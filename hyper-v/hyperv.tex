\documentclass[a4paper,12pt]{article}
\usepackage[T1]{fontenc}
\usepackage{imakeidx}
\usepackage{graphicx}
%\makeindex[columns=3, title=Alphabetical Index, intoc]

\begin{document}

\textbf{hyper-V}

\tableofcontents
\clearpage
 
\section{Introduction}
Here we spend some words on the virtualization topic

\section{How many we have out there?}

Here follows a list of the virtualization tools available on the internet to download and get to work with

\begin{itemize}
\item Hyper-V
\item VMWare
\item VirtualBox
\item Qemu
\end{itemize}

\section{Hyper-V}
Here we focus specifically on hyper-V

\subsection {What it is}
Hyper-V is a \emph{hypervisor-based} virtualization technology. Hyper-V uses the Windows hypervisor, which requires a physical processor with specific features.
In most cases, the hypervisor manages the interactions between the hardware and the virtual machines.

\subsubsection {hypervisor}

A hypervisor is computer software or hardware that enables you to host multiple virtual machines able to run
its own programs working optimally on a single piece of computer hardware.

\subsubsection {PROs and CONs}

Might not be relatively easy to install and set up especially on Windows 10 Home Edition.

You can have software running at a pretty decent performance as if it was not emulated at all

\subsection {Installation on Windows 10 Home Edition}
Unfortunately, the Hyper-V feature is only available in Windows 10 Professional and Enterprise editions. You can’t install it on Windows 10 Home Edition by default unless you go through some steps.

\subsubsection {Does your system support virtualization?}
Before moving forward, we need to check if our system supports virtualization.
Hardware virtualization is required for Hyper-V to function correctly.
Otherwise, you can use other virtualization platforms like Virtualbox and VMWare.

There are four basic requirements for Hyper-V to be installed on a Windows 10 computer:

\begin{itemize}
\item VM Monitor Mode Extensions
\item Virtualization enabled in firmware
\item Second level address translation
\item Data execution prevention
\end{itemize}

You can check all these requirements by opening the Command Prompt by running \emph{cmd} and running the systeminfo command. You can check the Hyper-V requirements section.

\subsubsection{Enable Hyper-V in Windows 10 Home}
You need to run "optionalfeatures" and tick the \emph{Hyper-V Management Tools} and \emph{Hyper-V Platform}

\section{network tester}

\clearpage

\printindex

\end{document}