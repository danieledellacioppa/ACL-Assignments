\documentclass[a4paper,12pt]{article}
\usepackage[T1]{fontenc}
\usepackage{imakeidx}
\makeindex[columns=3, title=Alphabetical Index, intoc]

\begin{document}
\tableofcontents

\section{Introduction}
In this document,we explain several parts \index{Introduction} of the Personal Computer
and every one of them should appear in the Index\index{Introduction} above.

\clearpage

\section{Central Processor Unit}
This section\index{CPU, Processor} is meant to illustrate the main component of a PC 
which is the Processor usually addressed as CPU.
This is the piece of hardware that phisically executes \emph{software} instructions one at at time.
Without this component there is no chance to have a running computer. It belongs to the bare minimum set of 
components that you need to have if you want to have something that you can call a PC.

Inside of it there is a set of \emph{registers} available for software developers to use.
Usually software developer write their code in a \emph{high level language} such as C++, Python, Java, etc.
The reason they're called high level languages is because there is a high level of abstraction from the processor
language level which is usually called assembly language (like 8086 or Motorola M68000)

Every \emph{core} of the processor executes one piece of code properly \emph {compiled} in the low level architecture
language by the compiler.

The reason why today we have usually more than one core is to allow concurrency which means more than one piece of
code executed at a time. This is achieved by developers by doing concurrent programming which essentially is all about
writing a software thinking at every instruction which process is executing it, so we'll find in the code checks like
\[if (pid==0)\]
where $pid$ is the ID of the $\bf process$ executing the "if" instruction which returns true only when the $\bf process$ hasn't  got any \emph{sons}. If it returns false then we're facing the father process, which usually is the process where the son has been generated from but
we'll go more in depht on this later on

%Let $D$ be a subset of $\bf R$ and let
%$f \colon D \to \mathbf{R}$ be a real-valued function on
%$D$. The function $f$ is said to be \emph{continuous} on
%$D$ if, for all $\epsilon > 0$ and for all $x \in D$,
%there exists some $\delta > 0$ (which may depend on $x$)
%such that if $y \in D$ satisfies
%\[ |y - x| < \delta \]
%then
%\[ |f(y) - f(x)| < \epsilon. \]

%One may readily verify that if $f$ and $g$ are continuous
%functions on $D$ then the functions $f+g$, $f-g$ and
%$f.g$ are continuous. If in addition $g$ is everywhere
%non-zero then $f/g$ is continuous.

\clearpage

\section{Hard Drive}
This is where data\index{data} is stored
\subsection{types of hard drive}
There are SSD\index{SSD} drives and traditional drives
\clearpage

\section{Memory}

This is where \emph{programs} allocate space for their \emph{execution}

\subsection{types of Memory}

We have SDRAM\index{SDRAM}, SO-DIMM\index{SO-DIMM}
\printindex

\end{document}
