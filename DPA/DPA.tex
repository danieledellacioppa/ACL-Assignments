\documentclass[a4paper,12pt]{article}
\usepackage[T1]{fontenc}
\usepackage{imakeidx}
\usepackage{graphicx}
%\makeindex[columns=3, title=Alphabetical Index, intoc]

\begin{document}

\textbf{Data Protection Act}


\tableofcontents
\clearpage

 
\section{Introduction}

Digital technology has transformed almost every aspect of our lives in the twenty years since the last Data Protection Act was passed.

Since information systems and business models become more
complex, a number of organisations may be working together
in an initiative that involves processing personal data.

\subsection{What it is}

makes our data protection laws fit for the digital age in which an ever increasing amount of data is being processed
empowers people to take control of their data
supports UK businesses and organisations through the change
ensures that the UK is prepared for the future after we have left the EU

 
\subsection{Why we have it}
 We use it to deal with data protection when dealing with Britsh data who doesn't belong to EU
 
When it comes to EU we use \textbf{GDPR} which is essentially DPA but applied when dealing with information coming from EU
\clearpage

\section{The key principles}
It is based around eight
principles of good information handling. These give people
specific rights in relation to their personal information and
place certain obligations on those organisations that are
responsible for processing it.

\subsection {Lawfulness, fairness and transparency}
Ensures that users can understand what it is there are signing up to when they hand over personal data. This principle requires that organisations use language that is ‘clear, plain and accurate’ as to what a data subject is consenting to
\subsection{Purpose limitation}
Stipulates that personal data, which is collected for a specific, previously stated and understood purpose, must not then be used for other applications. 
\subsection{Data minimisation}
Ensuring that the extent or amount of data collected and/or processed is adequate, relevant and limited to the intended purpose
\subsection{Accuracy}
Makes organisation responsible for either updating inaccurate information or getting rid of it.
\subsection{Storage limitation}
Restricts organisations from keeping hold of data for indefinite periods of time, or beyond that of its intended purpose
\subsection{Integrity and confidentiality}
Previously known as the ‘security’ principle, integrity and confidentiality of personal data must be upheld with the appropriate security measures. As with many of the other principles, there is an inherent responsibility to implement both physical and technological controls to ensure compliance.
\subsection{Accountability}
Requires organisations to take responsibility for the personal data being handled and their compliance with the other six principles. Appropriate measures and records are also required to be in place as to demonstrate compliance.

\subsection{International Transfer of Data}
Regulates the transfer of personal data to countries or organisations outside the UK to make sure they are sufficiently compliant with the standards laid forth by the legislation.

\clearpage
 \section{Difference between data processor and data controller}
The DPA draws a distinction between a ‘data controller’ and a
‘data processor’ in order to recognise that not all organisations
involved in the processing of personal data have the same
degree of responsibility. It is the data controller that must
exercise control over the processing and carry data protection
responsibility for it. 

\subsection{data controller}
“data controller” means a person who (either alone or
jointly or in common with other persons) determines the
purposes for which and the manner in which any personal data
are, or are to be processed

\subsection{data processor}
“data processor”, in relation to personal data, means any
person (other than an employee of the data controller) who
processes the data on behalf of the data controller.

\subsection{processing}
It means
obtaining, recording or holding data or
carrying any set of operations on the
data, including
\begin{itemize}
    \item organisation, adaptation or alteration of the information or data
    \item retrieval, consultation or use of the information or data
    \item  disclosure of the information or data by transmission dissemination or otherwise making available
    \item  alignment, combination, blocking, erasure or destruction of the information or data
\end{itemize}

\section{GDPR penalties}

There will be two levels of fines based on the GDPR.  The first is up to €10 million or 2\% of the company’s global annual turnover of the previous financial year, whichever is higher.  The second is up to €20 million or 4\% of the company’s global annual turnover of the previous financial year, whichever is higher.  The potential fines are substantial and a good reason for companies to ensure compliance with the Regulation.

The Parliament had requested for fines to reach €100 million or 5\% of the company’s global annual turnover.  The agreed fines are the compromise that was reached.

Fines for infringements will be considered on a case-by-case basis and will take a number of criteria into consideration, such as the intentional nature of the infringement, how many subjects were affected and any previous infringements by the controller or processor.

\clearpage

\printindex

\end{document}
