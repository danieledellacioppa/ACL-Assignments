\documentclass[a4paper,12pt]{article}
\usepackage[T1]{fontenc}
\usepackage{imakeidx}
\usepackage{graphicx}
%\makeindex[columns=3, title=Alphabetical Index, intoc]

\begin{document}

\textbf{IT Incident Log}


\tableofcontents
\clearpage

\Large
\section{Brief description}
\normalsize
In this part of the document,it is explained briefly the \index{incident} happened, why it happened, how many problems and errors it caused.

\Large
\subsection{Root Cause}
\normalsize
This section goes more in depht on the description of the reason why the incident has happened. Where did we failed and essentially gives a hint to reduce the likelyhood for that particular incident to happen again

\Large
\subsubsection{Corrective and Preventive measures}
\normalsize
This is where it is explained deeply how the right behaviour needs to looks like in order to lower the risk to face the same incident again. It is essentially a list of best practices to follow to face the particular incident taken in exam. These best practices have been designed after reading the Root Cause with careful consideration

\Large
\section{Timeline}
\normalsize
This section\index{Timeline, time} is meant to show when every single event happened as well as the actions takes to resolve it and how long did it take to fix it

\Large
\subsection{Resolution and recovery}
\normalsize
This section goes more in depth about the set of actions take to deal with every single event related to the incident providing a deeper description of the procedure followed to face the single event
\clearpage


\Large
\section{Evidences}
\normalsize
This section includes pictures, audio recordings and any media like videos related to the incident that we have been given permission to include in the file

There is no limit to the size of pictures to make sure we can address as many details as we possibly can like in the example:


\includegraphics[width=15cm]{./boris-pintoflager.jpeg}

\Large
\section{Financial Impact}
\normalsize
This section lists the expenses that could not be controlled due to the incident, drafting a hint of plan to try to recover the costs in the shorter future possible

\Large
\subsection {Financial Recovery Plan}
\normalsize
This is where the Financial recovery plan is explained. It contains the strategy to manage to stay in budget over a limited amount of time

\printindex

\end{document}
