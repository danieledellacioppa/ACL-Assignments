\documentclass[a4paper,12pt]{article}
\usepackage[T1]{fontenc}
\usepackage{imakeidx}
\usepackage{graphicx}
%\makeindex[columns=3, title=Alphabetical Index, intoc]

\begin{document}

\textbf{Computer Laboratory ITIS Giordani Caserta Incident Log}


\tableofcontents
\clearpage

\Large
\section{Brief description}
\normalsize
In this part of the document,it is explained briefly \emph{what} \index{incident} happened, \emph{why} it happened and how many problems and errors it caused on the setups in the Lab.

\Large
\subsection{Root Cause}
\normalsize
This section goes more in depht on the description of the reasons why the incident has happened. Where did the teacher or the class assistant failed and essentially gives a hint to reduce the likelyhood for that particular incident to happen again

\Large
\subsubsection{Corrective and Preventive measures}
\normalsize
This is where it is explained deeply how the right behaviour needs to look like in order to lower the risks to face the same incident again. It is essentially a list of \emph{best practices} to follow to face the particular incident taken in exam. These \emph{best practices} have been designed after reading the Root Cause with careful consideration

\Large
\section{Timeline}
\normalsize
This section\index{Timeline, time} is meant to show when every single event happened as well as the actions taken to solve it, how long did it take to fix it and how many technical class assitant did we need in order to fix it promptly.

\Large
\subsection{Resolution and recovery}
\normalsize
This section goes more in depth about the set of actions to take to deal with every single event related to the incident providing a deeper description of the procedure followed to face the single event
\clearpage


\Large
\section{Evidences}
\normalsize
This section includes pictures, audio recordings and any media like videos related to the incident that we have been given permission to include in the file

There is no limit to the size of pictures to make sure we can address as many details as we possibly can like in the example:


\includegraphics[width=15cm]{./boris-pintoflager.jpeg}

\Large
\section{Financial Impact}
\normalsize
This section lists the expenses (if there are any) that could not be controlled due to the incident, drafting a hint of plan to try to recover the costs in the shorter future possible

\Large
\section{Psychological Impact}
\normalsize
This section covers the psyichological impact on students who were present where the incident has happend. How do they feel, is there anything we have to do for them to make them feel safe and protected in school, how much do they feel they can trust teachers and technical laboratory assistants who are in charge at the time of writing and responsible for the whole building, and adds as well telefone contact and email belonging to teachers and assistants present at the time of the incident.


\Large
\subsection {Financial Recovery Plan}
\normalsize
This is where the Financial recovery plan is explained (if needed). It contains the strategy to manage to stay in budget over a limited amount of time


\printindex

\end{document}
