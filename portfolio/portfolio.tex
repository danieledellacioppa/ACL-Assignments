\documentclass[a4paper,12pt]{article}
\usepackage[T1]{fontenc}
\usepackage{imakeidx}
\usepackage{graphicx}
%\makeindex[columns=3, title=Alphabetical Index, intoc]
\usepackage{graphicx,wrapfig,lipsum}

\begin{document}

\textbf{Daniele Della Cioppa}

\textbf{Software Developer}

\tableofcontents
\clearpage

\section{Knowledge unit}

This is what I've been covering so far during the apprenticeship:

\begin{itemize}
\item {database development}
\item {application life cycle}
\item docker
\item github 
\item LateX
\item PostgreSQL
\item Linux
\item {iOS \& Android development}
\end{itemize}
\clearpage

\subsection{iOS \& Android development}
In Figure 1 we can see the stage I was on the 23rd March, with the mobile development since this is being literally self taught

\begin{wrapfigure}{r}{5cm}
%\centering
\includegraphics[width=5cm]{./capture-app.PNG}
%\vspace{-10pt}
\caption{Hello World in Xamarin with a button}\label{wrap-fig:1}
\end{wrapfigure} 


We realized Xamarin brings in some difficulties even experts are struggling with. Since I'm on an apprenticeship we agreed to make it simple and use Android Studio to make an app for Android and later on use XCode for Apple. The problem with Apple is we'll need a real physical iPhone to test the app on

After a little research I found out a tool called maptiler but it was going to be sunset soon so I've had to search more and I came up with \textbf{OSMDroid}. The following are some screenshots of how the app is looking like. Some functionalities have been added to test separately Client and Server functions. My app needs to interact with a C++ Server but because of the burden it takes to switch the real server ON I've developed a quick server to have tests with and then once in a while see how it goes with the real server when it gets turned on

\clearpage

In figure 2 you can find the stage of the 5th of May where we're experimenting talking client vs server. The functionality it's still incapsulated in the client and needs to become part of the app rather than an external button
\begin{wrapfigure}{r}{5cm}
%\centering
\includegraphics[width=3cm]{./current_status_g7.PNG}
\includegraphics[width=3cm]{./server_g7.PNG}
%\vspace{-10pt}
\caption{stage I was at, on the 23rd March}\label{wrap-fig:2}
\end{wrapfigure}


\includegraphics[width=3cm]{./current_status_g8.PNG}
\includegraphics[width=3cm]{./client_g8.PNG}

\section{Introduction}

After three months in my role I'm now responsible for the following tasks with databases:

\begin{itemize}
\item {reading the specs}
\item design
\item implementation
\item testing 
\item documenting
\end{itemize}

\clearpage

\subsection{design}
This is a first idea of the database I came up with

\noindent \includegraphics[width=14cm]{./ERSchemaGen2.jpg}

We agreed it was holding probably more information than what we need to display in the app so I came up later on with a smaller design which really has just the right amount of things we need to make sure we can run enough tests and see how the App, the Server and the Database interact between each other
\clearpage

This is the smaller database designed again from scratch. We need to add geolocation to nodes but for the time being it will be hard coded in the server and we'll try to see if we can display a hardcoded geolocation hold by the server to be caught by the client and displayed properly on the app

\noindent \includegraphics[width=14cm]{./SecondERSchemaGen2.jpg}

\printindex

\end{document}
