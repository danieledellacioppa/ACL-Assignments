\documentclass[a4paper,12pt]{article}
\usepackage[T1]{fontenc}
\usepackage{imakeidx}
\usepackage{graphicx}
%\usepackage{hyperref}

\makeindex[columns=3, title=Alphabetical Index, intoc]


\usepackage{listings}
\usepackage{color}

\definecolor{dkgreen}{rgb}{0,0.6,0}
\definecolor{gray}{rgb}{0.5,0.5,0.5}
\definecolor{mauve}{rgb}{0.58,0,0.82}

\lstset{frame=tb,
  language=Java,
  aboveskip=3mm,
  belowskip=3mm,
  showstringspaces=false,
  columns=flexible,
  basicstyle={\small\ttfamily},
  numbers=none,
  numberstyle=\tiny\color{gray},
  keywordstyle=\color{blue},
  commentstyle=\color{dkgreen},
  stringstyle=\color{mauve},
  breaklines=true,
  breakatwhitespace=true,
  tabsize=3
}



\begin{document}
\textbf{archlinux postgres installation}
\tableofcontents

\section{Introduction}

In this document we are going through all the steps I historically needed to install postgres on ArchLinux.


\section{Installation}

\subsection {package installation}
\[sudo \quad pacman -S \quad postgresql \]


\begin{lstlisting}
  resolving dependencies...
  looking for conflicting packages...   
  Packages (2) postgresql-libs-13.4-6  postgresql-13.4-6   
  Total Download Size:   18.36 MiB   
  Total Installed Size:  59.23 MiB   
  :: Proceed with installation? [Y/n]         
\end{lstlisting}

\subsection {changing postgres password}

\[sudo \quad passwd \quad postgres  \]

then change the password

\[sudo \quad -iu \quad postgres \]


\[ psql \]

\begin{lstlisting}
psql: error: could not connect to server: No such file or directory  
Is the server running locally and accepting   
connections on Unix domain socket "/run/postgresql/.s.PGSQL.5432"?   
\end{lstlisting}

\subsection {initialize the database cluster}

let us initialize the database cluster by issuing this command

\[ initdb \quad -D \quad /var/lib/postgres/data \]

\begin{lstlisting}
  [postgres@daniele-virtualbox ~]$ initdb -D /var/lib/postgres/data  
  The files belonging to this database system will be owned by user "postgres".   
  This user must also own the server process.   
  The database cluster will be initialized with locales   
  COLLATE:  en_GB.utf8  
  CTYPE:    en_GB.utf8  
  MESSAGES: en_GB.utf8  
  MONETARY: en_GB.UTF-8  
  NUMERIC:  en_GB.UTF-8  
  TIME:     en_GB.UTF-8  
  The default database encoding has accordingly been set to "UTF8".  
  The default text search configuration will be set to "english".   
  Data page checksums are disabled.   
  fixing permissions on existing directory /var/lib/postgres/data ... ok   
  creating subdirectories ... ok   
  selecting dynamic shared memory implementation ... posix   
  selecting default max_connections ... 100   
  selecting default shared_buffers ... 128MB   
  selecting default time zone ... Europe/London   
  creating configuration files ... ok   
  running bootstrap script ... ok   
  performing post-bootstrap initialization ... ok   
  syncing data to disk ... ok   
  initdb: warning: enabling "trust" authentication for local connections   
  You can change this by editing pg_hba.conf or using the option -A, or   
  --auth-local and --auth-host, the next time you run initdb.   
  Success. You can now start the database server using:   
  pg_ctl -D /var/lib/postgres/data -l logfile start   
\end{lstlisting}




as you can see all is being done from the postgres user as it will be the only user meant to be using the database

\[ pg\_ctl \quad -D \quad /var/lib/postgres/data \quad -l \quad logfile \quad start \]


\begin{lstlisting}
  waiting for server to start..../bin/sh: line 1: logfile: Permission denied   
  stopped waiting   
  pg_ctl: could not start server   
  Examine the log output.   
\end{lstlisting}

\subsubsection{ granting root privileges to postgres user}

let us make postgres a sudoer by adding it via the command `sudo visudo`. Once that's done then issue the following command:

\subsubsection{starting the postgresql service}
\[ systemctl \quad status \quad postgresql.service \]

\begin{lstlisting}
     postgresql.service - PostgreSQL database server  
     Loaded: loaded (/usr/lib/systemd/system/postgresql.service; disabled; vendor preset: disabled)  
     Active: inactive (dead)   
\end{lstlisting}


the following command then gets issued from the postgres user

\[ initdb \quad --locale=en\_US.UTF-8 \quad -E \quad UTF8 \quad -D \quad /var/lib/postgres/data \]

\begin{lstlisting}
  initdb: error: directory "/var/lib/postgres/data" exists but is not empty   
  If you want to create a new database system, either remove or empty   
  the directory "/var/lib/postgres/data" or run initdb   
  with an argument other than "/var/lib/postgres/data".   
\end{lstlisting}

\subsubsection{ housecleaning }

ideally i'll try to remove or empty the directory mentioned above

\[sudo \quad mkdir \quad /var/lib/postgres/data \]

\[sudo \quad chown \quad postgres:postgres \quad /var/lib/postgres/data \] 

\[sudo \quad -u \quad postgres \quad initdb \quad -D \quad /var/lib/postgres/data \]


\begin{lstlisting}

  mkdir: cannot create directory ‘/var/lib/postgres/data’: File exists   
  could not change directory to "/home/daniele": Permission denied   
  The files belonging to this database system will be owned by user "postgres".   
  This user must also own the server process.   
  The database cluster will be initialized with locales   
  COLLATE:  en_GB.utf8  
  CTYPE:    en_GB.utf8  
  MESSAGES: en_GB.utf8  
  MONETARY: en_GB.UTF-8  
  NUMERIC:  en_GB.UTF-8  
  TIME:     en_GB.UTF-8  
  The default database encoding has accordingly been set to "UTF8".  
  The default text search configuration will be set to "english".   
  Data page checksums are disabled.   
  initdb: error: directory "/var/lib/postgres/data" exists but is not empty  
  If you want to create a new database system, either remove or empty   
  the directory "/var/lib/postgres/data" or run initdb   
  with an argument other than "/var/lib/postgres/data".   
\end{lstlisting}



\section { We are good to go }

Here I took some steps:

\begin{enumerate}
  \item I removed the data folder with sudo privileges
  \item I added postgres to the sudoers and rebooted
  \item then issued the following commands again
\end{enumerate}


\[ sudo \quad mkdir \quad /var/lib/postgres/data \]
\[ sudo \quad chown \quad postgres:postgres \quad /var/lib/postgres/data \]
\[ sudo \quad -u \quad postgres \quad initdb \quad -D \quad /var/lib/postgres/data \]
\[ postgresql-setup --initdb \]
\[ systemctl restart postgresql.service \]
\[ systemctl enable postgresql.service \]

and it all started to work but I've made another simpler guide which I'll be releasing shortly

\printindex



\end{document}

